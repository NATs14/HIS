\documentclass[a4j,12pt,twoside]{jreport}
\usepackage{geometry}
\usepackage[dvipdfmx]{graphicx}
\usepackage{multicol}
\usepackage{float}
\usepackage{arydshln}
\usepackage{kit-is-b}
\author{福田 遼}
\idnumber{12122049}
\deadline{2016}{2}{15} % 提出日{西暦年}{月}{日}
%% 長いタイトルで途中改行する場合,改行したい位置で
%% \coverbreak, \abstbreak, \\ のいずれかを使う.
%%   \coverbreak :表紙でのみこの位置で改行
%%    \abstbreak :概要でのみこの位置で改行
%%            \\ :表紙・概要ともにこの位置で改行
%%【例】
%% \title{長い長いタイトル\coverbreak の改行位置\abstbreak に関する研究}
\title{スマート家電との音声対話における\\
発話表現への特徴付けによる家電の判別性向上手法}
\advisor{1}{辻野 嘉宏} % 指導教員(1:教授,2:准教授)
\secondadvisor{2}{倉本 到} % (必要ならば)指導教員(1:教授,2:准教授,4:助教)
\thirdadvisor{4}{山本 景子} % (必要ならば)指導教員(1:教授,2:准教授,4:助教)

\begin{document}
\maketitle

\begin{abstract}

スマート家電が連携して動作する場合,ユーザは家電の動作が自身の要求に合わなかったとき,手動で自身の望み通りに家電を設定し直せる.
しかし,既存のスマート家電では現在の家電の状態の通知が不十分であるため,ユーザは家電の状態をそれぞれ確認する必要があり,非効率である.

そこで,家電の状態を家電自身に発話させ,家電の発話に口調および声色の特徴を付与することでユーザが早く正確に発話した家電を判別できる手法を提案する.
そして,この提案手法の評価実験を行った.
実験の結果,提案手法は:発話した家電を正確に判別でき,作業負荷を減らせることがわかった.
\end{abstract}

\begin{contents}
\tableofcontents % 目次の作成
\end{contents}

\chapter{緒言}
家の中には,テレビ,エアコン,空気清浄機,照明など多くの家電がある.
これら全ての家電の状態を把握し,家電をそれぞれ設定してユーザが自身の望む状態にするには多くの労力と時間を必要とする.

数多くの家電をユーザの望む設定にする労力を軽減するための方法に,設定の自動化がある.
設定を自動化するために考案されたのが,スマート家電やホームオートメーションシステムである.
スマート家電とはネットワークに接続されている家電のことであり,ホームオートメーションとはスマート家電を用いて家全体の家電の設定の管理を自動化することである.
これらに関する研究や議論はこれまでに数多くなされており\cite{Grinter}\cite{TRON},ECHONET\cite{ECHO}というスマート家電専用の通信規格が整えられている.

既存のホームオートメーションシステムは次の3機能を持っており,主にタブレット端末上で操作される.
\begin{description}
	\item [ 機能(a)]全家電が現在どのように動作しているかを一覧表示する機能
	\item [ 機能(b)]各家電を操作できる機能
	\item [ 機能(c)]複数の家電を自動的に操作できる機能
\end{description}
機能(a)は,端末上に全ての家電あるいはユーザが選択した家電の状態を一覧表示する機能である.
機能(b)は,ユーザが端末上で操作したい家電を選択し,どのような操作を行わせるかをユーザが設定できる機能である.
また,ホームオートメーション機能のメリットである家電の設定の自動化は機能(c)によって行われる.
これは,ユーザが予め一度設定しておけば,その後ユーザが特定の操作を行うことで複数の家電をその設定していた通りの状態に自動的に変更するという機能である.
例えば,『スポーツ観戦モード』というボタンを押すと,テレビ,カーテン,照明,スピーカなどの設定が,スポーツ観戦に適した設定に自動で変更される.

しかし,これらのホームオートメーション機能を実行するには,ユーザが望む環境には含まれないタブレット端末を操作しなければならない.
例えば,「サッカーを見たい」と思ったとき,ユーザはテレビの電源を入れようとする.
しかし,ホームオートメーションシステムの機能を用いる場合,ユーザはテレビとは直接関係のないタブレット端末を操作し『スポーツ観戦モード』を選択しなければならない.
選択肢の量が少なければ選択するのに時間はかからないが,ホームオートメーションシステムの機能で設定する家電の数が多い場合は,ユーザビリティを低下させる原因となる.

この問題を解決するために,複数の家電が連携して動作するシステムが提案されている\cite{denshin}\cite{Hitach}.
これらの家電連携システムはホームオートメーションを実現するシステムであり,ある家電の状態が変わると,それに応じてユーザの要求に合うように他の家電が自動的に状態を変える.
既存のホームオートメーションシステムにこれらの家電連携システムを用いると,サッカーを見たいと思ったときにテレビの電源をつけるだけで,前述の『スポーツ観戦モード』のボタンを選択するときと同じようにスピーカなどの家電を要求通りの状態に変えられる.

しかし,これらの家電連携システムには問題が2点ある.1点目は,家電がユーザの要求にそぐわない動作を行う可能性があることである.
例えば,毎朝ホットコーヒーを飲む人はこのシステムを用いて,朝起きたときにホットコーヒーを飲めるように,目覚ましを止めたときにコーヒーメーカーが連携してホットコーヒーを入れる動作をするように設定する.しかし,日によってはアイスコーヒーを飲みたいと考えることが起こりうる.
このことから,ユーザの要求は必ずしも家電が自動で行う動作で満たされるとは限らない.

2点目は,家電の動作が連携によりいつどのように変わったかをユーザが把握しにくいことである.
家電の動作が変わったことで実現された環境がユーザの要求に合っていなかったとき,ユーザは要求に合うように家電の状態を手動で設定し直したいと考える可能性がある.
しかし,それを実現するためには,ユーザが家電の状態を把握する,または家電連携システムが現在の家電の状態をユーザにわかりやすく通知する必要がある.

以上の2点の問題を解決するために,筆者の所属する研究グループでは,スマート家電が状態が変化していることをユーザに音声でわかりやすく通知し,通知内容がユーザの要求と異なる場合,ユーザが家電の状態を自身の要求に合うように音声で修正できる手法(以降,先行手法\cite{sumi})を提案している.
先行手法では,ユーザに家電が何の動作をしようとしているかを知らせるための「動作内容」と,なぜその動作を行うかを知らせるための「動作理由」を発話する.
そして,先行システムと既存のホームオートメーションシステムのプロトタイプとの比較実験の結果,家電の状態が変わるときに「動作内容」と「動作理由」の2点を単に家電に発話させるだけでは,ユーザはどの家電が発話したかを判別しにくいことがわかっている.
さらに,どの家電が発話したかの判断に迷っている間に家電が状態を変更してしまうという問題があることがわかっている.
そこで本報告書では,この先行システムにおいて家電の状態変化の通知をユーザが判別しにくいという問題を解決することを目的とし,家電の音声にそれぞれ特徴を持たせる手法を提案する.

以降,2.では先行システムとその問題点をまとめ,3.ではその問題点を解決する提案手法を述べる.4.では提案手法の有用性を確かめるために,提案手法を用いたプロトタイプと先行システムを模擬したプロトタイプを比較する評価実験について述べる.

\chapter{先行研究}

\section{定義}
先行研究において以下の用語が定義されている.本報告書でもこの定義を用いる.
\begin{itemize}
\setlength{\leftskip}{1.72cm}
	\item [{\bf 連携}   ]家電Aの状態が変化し,それにより家電Bの状態が変化するとき,BはAに連携しているという
	\item[{\bf 要修正家電}]連携により変化した家電のうち,その変化の内容がユーザの要求に合っていない家電のこと
	\item[{\bf 待機家電} ]家電A, Bが存在し,AにBが連携しているとする.
このとき,Aの状態が変化し,Bが実際に状態を変化させるまでの間の状態であるBのこと
	\item[{\bf 介入}   ]待機家電に対して,ユーザが特定の行動を取ることによって,待機家電に変化中止や変化の内容の変更等の何らかの動作をさせること
\end{itemize}

%\begin{description}
%\setlength{\leftskip}{0.4cm}
%	\item [連携]家電Aの状態が変化し,それにより家電Bの状態が変化するとき,BはAに連携しているという
%	\item[要修正家電]連携により変化した家電のうち,その変化の内容がユーザの要求に合っていない家電のこと
%	\item[待機家電]家電A,Bが存在し,AにBが連携しているとする.
%このとき,Aの状態が変化し,Bが実際に状態を変化させるまでの間の状態であるBのこと
%	\item[介入]待機家電に対して,ユーザが特定の行動を取ることによって,待機家電に変化のキャンセルや変化の内容の変更等の何らかの動作をさせること
%\end{description}
\section{概要}
ユーザの望んでいる環境が設定された動作により実現される環境と異なる場合,要修正家電が発生する.
これらの要修正家電の状態をユーザの要求に合うように設定し直すには,まずユーザは要修正家電の存在に気付く必要があるが,既存のホームオートメーションシステムではユーザは端末の画面を見ない限り気付けない.
また,ユーザが端末の画面を見ていたとしても,ユーザが知ることができるのは各家電の現在の状態だけである.
どの家電でどのような変化が発生してきたかという通知は行われないため,ユーザが設定し直す必要があるかを確認するには,常に各家電の状態を確認する必要があり非効率である.

以上の既存のホームオートメーションシステムの問題点を以下にまとめる.
\begin{description}
	\item [ 問題(A)]ユーザの要求が変わると要修正家電が発生するが,すぐに対応できない
	\item [ 問題(B)]どの家電でどのような変化が発生したかをユーザが把握しづらい
\end{description}

これらの問題を解決するために,先行研究では音声対話に基づき連携している家電とインタラクションする手法を提案している.この手法における家電の操作の流れを以下に示す.介入を可能にすることで問題(A)を解決することを目指し,明示的な通知を行うことで問題(B)を解決することを目指す.
\begin{enumerate}
	\item ユーザの操作によって,ある家電の状態が変化する
	\item 状態が変化した家電が自分の変化の内容を発話する
	\item 状態が変化した家電に連携している別の家電の状態が変化する
	\item 変化の内容と理由をユーザに発話する
	\item 変化の内容がユーザの望む動作でない場合,ユーザが介入を行う
	\item ユーザの介入内容に応じて家電が変化する
\end{enumerate}
\section{家電との対話}
\label{sec:対話}
問題(A)で示したように,既存のホームオートメーションシステムではその時々のユーザの要求をシステムが推定することが困難であるため,自動的に変化が発生する家電全てが要修正家電となる可能性がある.そのため,家電が待機家電となったときに,
\begin{itemize}
	\item 待機家電となった原因
	\item どのように状態を変えるのか
\end{itemize}
の2点を家電に発話させることでユーザに変化が発生しようとしていることを通知する.
例えば,ユーザが暖房をつけると暖房に連携している空気清浄機が音声で「暖房がONになったので電源をONにします」という通知を行う.
このようにすることで,ユーザはどのような連携が起こり,要修正家電が発生しようとしているかを知ることができる.

これによりユーザは家電の発話内容がユーザの要求にそぐわなかった場合,介入を行うことで要修正家電の発生を阻止できる.一方,ユーザの要求に合っていれば介入を行う必要はないため,一定時間経過後に家電が自動で動作を始める.

また,家電は連携しているため,連携の先頭の方の家電が要修正家電であった場合,その家電に介入を行いユーザの要求に合う動作にすることで,その家電に連携している家電全ての動作を要求に合うようにできる.

家電が発話し始めた直後にユーザが介入操作を行うことは困難である.
介入を行うかどうかの判断と介入の操作を行うのはユーザであるため,そのための時間猶予が必要である.猶予を持たせるには,家電が通知を行ってから変化が発生するまでに一定のタイムラグを持たせる方法と,通知された変化の発生をユーザに承認させる方法の2通りが考えられる.しかし,後者では家電の通知の度にユーザの操作を必要としてしまうため,日常の生活の邪魔になってしまう可能性がある.一方,前者では,介入を行わない場合はユーザは家電の操作に集中する必要がなくなるため,ユーザ自身が主に行っている作業に集中できると考えられる.
以上のことから,先行システムでは前者を採用し,ユーザの介入可能時間を8秒としている.

\section{問題点}
\label{sec:問題}
先行システムは家電の口調および声色が全て同じであり,発話した家電の判別が困難であるため,発話している家電の方向からどの家電が発話したかをユーザが判別しなければならない.
そのため,家電がユーザから見て同じ方向に複数存在していた場合,家電の判別は困難である.
また,\ref{sec:対話}で述べたように家電が発話を行ってから変化が発生するまでには時間的猶予が8秒あるが,
ユーザが家電の発話後にどの家電が発話したかの判別に時間がかかると,ユーザが介入するか否かを判断する時間が短くなってしまう.最悪の場合,ユーザが介入すべきかどうかを考えているうちに次の家電の動作が発生してしまう場合がある.
しかも,ユーザは発話した家電を判別するために音声が聞こえた方向を確認しなければならない,これでは,
音声によって通知を行っているにも関わらず,ユーザの視覚を必要としてしまうため,ユーザが視覚的注意を向けて行っている作業を妨げてしまう.
したがって,ユーザが作業をしながらでも音声によって家電の状態を把握できるというメリットがなくなってしまう.
このメリットを活かすためには,ユーザが視覚的注意を向けなくても発話した家電を判別できるような音声を用いての通知を行う必要がある.

\chapter{提案手法} 

\section{発話特徴による判別性の向上}

\ref{sec:問題}で示した家電の判別が困難であるという問題に対処するために,家電が動作の内容と理由を音声により通知する際に,家電それぞれに異なる口調や声色の発話表現の特徴を持たせることで,ユーザにどの家電が発話したかと発話の内容を通知する手法を提案する.
このようにすることで,ユーザはその特徴の違いにより,早く正確にどの家電が発話したかを判別できる.

これに対し,単に発話前に家電に家電名を発話させれば,家電に特徴を持たせるよりも確実にユーザが家電を判別できると考えられる.しかし,その方法では,ユーザが作業に集中している場合に家電名を聞き逃してしまう可能性がある.

また,家電名を発話させると,連携する家電の数が増えた場合にその分だけ家電の発話時間が長くなってしまい,ユーザが家電の発話に意識を向けなければならない時間が長くなってしまう.
そのため,結局ユーザの集中を妨げてしまう可能性がある.
しかし,声色の違いによる判別であれば,ユーザが家電の発話の言葉を聞き取れない場合でも,声色の違いを聞き分けることにより,発話した家電を推測できると考えられる.
さらに,ユーザが各家電の音声の特徴を把握していれば,家電の発話ごとに家電に家電名を言わせる必要がなくなり,家電の発話時間を抑えることもできる.
\section{シナリオ}
提案手法を導入したときの事例を以下に示す.A君は一人暮らしをしている男性である.
\begin{quote}

 一人暮らしをしているA君は,夕飯の食材の買い出しから帰宅した.A君は荷物で両手が塞がっていたため,「照明つけて」と発話すると,照明が「電源をONにするよ」と発話し,部屋の照明がついた.
また,空気がこもっていたため,空気を入れ換えようと思い「窓開けて」と発話した.すると,窓が「開けんで」と発話し,窓が開いた.
それから夕食を作り始めた.夕飯の食材を炒めるためにコンロに火をつけると,換気扇が「コンロがついたので換気扇を回します」と発話し,自動的に換気扇が回り始めた.よって,A君は換気扇を操作しないで,換気扇を回すことができた.\\
 夕食が完成し,A君は夕食を食べ始めようと思ったが,窓を開けていたため部屋がとても寒かった.そこで,暖房をつけようと思い手動でエアコンをつけると,エアコンが「電源をONにしとくよ」と発話した.
すると,窓が「エアコンが ON になったから閉めんで」と発話し,自動的に閉まった.
そして,窓が閉まってから,空気清浄機が「窓が閉まったので起動しますね」と発話し,自動的に起動した.
さらに,換気扇が「エアコンが ON になったから止めます」と発話し,自動的に止まった.\\
 その後,A君は手動でこたつをつけた.するとこたつが「電源ONにしまっせ」と発話した.そして,A君は夕食を食べた.
A君は夕食を済ませ,風呂に入ったためそろそろ寝ようと思い,部屋の照明を消した.
すると,暗闇で「照明が OFF になったから消しまっせ」と聞こえた.その口調から,それがこたつの声だとA君にはわかり,A君はこたつをつけっぱなしだったことに気付いた.
さらに,「照明が OFF になったから消すよ」と聞こえた.その口調から,それがエアコンだと気付き,A君はエアコンをすぐに消してしまうと寒いと思ったため,A君が「1時間ついといて」と言うと,エアコンが「1時間つけたままにしとくよ」と発話した.

\end{quote}

\section{関連研究}
音声の特徴が発話者の個人性に与える影響を調べる研究は数多くなされている.
橋本ら\cite{bunseki}は音声が個人性の知覚に与える影響度の分析を行っており,それによると,スペクトル,基本周波数が個人性の知覚に顕著に影響しており,さらに,既知話者の場合は未知話者の場合に比べてスペクトルの影響が1.6倍大きいことがわかっている.
また,北村\cite{seisei}はスペクトルに含まれる高周波数帯域(1,740〜8,000 Hz)の違いが最も話者認識に影響を与えることを示しており,それに基づき物真似タレントの物真似音声の分析を行っている.その結果,物真似タレントの物真似音声と物真似対象者の音声のスペクトルの概形が似ている一方で,物真似タレントの地声音声と物真似音声のスペクトルの概形は異なるという結果になっている.
これらのことから,スペクトルや周波数は発話の特徴に大きく影響を与えることがわかる.
スペクトルは発話の特徴のうち,声色に影響を与えるものである.そのため,ユーザが家電の発話の特徴を把握できれば,声色の違いを家電の発話に持たせることによって家電の判別が容易になると考えられる.

宮崎ら\cite{chara}は言語表現によって対話エージェントの発話にキャラクタ性を持たせ,バリエーション豊かな言語表現を生成する手法を提案している.言語表現によって様々な年齢,男女の違いを持たせた表現によってキャラクタ性を作成し,被験者にその表現がどの年齢のどの性別かを答えさせることで適切なキャラクタ性を持たせられているかの評価を行った結果,キャラクタ性の判別性が約30\%向上できるという結果が得られている.このことから,口調の違いを家電に持たせることによっても家電の判別性が向上すると考えられる.
\chapter{評価実験}

\section{目的}
本実験の目的は,以下の通りである.
\begin{enumerate}
	\item 家電が発話するシステムに慣れたユーザが,提案手法で先行手法に比べて家電の判別のしやすさがどの程度改善されるかを確かめる
	\item 同じ方向に複数の家電が存在していても家電の判別が可能かを確かめる
	\item どの発話方法の組み合わせが最も使いやすいかを確かめる
\end{enumerate}

\section{実験方法}
\subsection{比較対象}
比較対象として,家電の連携時の発話について下記の手法a.〜d.の違いのある実験用システムを用いる.
\begin{description}
	\item[手法a.] 家電名を発話し,家電ごとに口調および声色の特徴がある
	\item[手法b.] 家電名を発話しないが,家電ごとに口調および声色の特徴がある(提案手法)
	\item[手法c.] 家電名を発話するが,家電ごとに口調および声色の特徴がない
	\item[手法d.] 家電名を発話せず,家電ごとに口調および声色の特徴がない(先行手法)
\end{description}
また,以下の部分は手法a.〜d.で共通である.
\begin{itemize}
	\item 機能\\
	本実験では提案手法を用いて家電の状態の判別を正確に行えるかを評価するため,タスクを「発話した家電と家電の状態の把握をすること」のみとする.そのため,ユーザの要求通りに家電の変化を修正する介入動作は実装されていない.
	\item 動作\\
	実験用システムでは,家電は以下のように動作する.
	\begin{enumerate}
		\item ユーザが音声操作によって家電を操作し,何らかの家電の状態を変化させる
		\item 何らかの家電Aが変化したとき,それに連携している家電Bが存在している場合,家電Bは自分の変化の内容および変化の理由を発話する
		\item 連携している最後の家電の発話が終了したとき,終了したことを知らせるビープ音が流れる
	\end{enumerate}
	\item 実装\\
	各家電が発話しているようにユーザが感じるようにするために,実験用システムではスピーカを家電に見立てる.また,これをPCと接続し,ユーザの操作に合わせて実験者がスピーカを制御する.
	\item 発話特徴\\
	家電6個(A〜F)に対し,手法a,bでは各家電ごとに異なる口調および声色を用い,家電の音声には機械音声ではなく,予め録音した男性4名,女性2名の計6名の人間の声を用いる.手法c,dでは先行研究と同様にするために機械音声を用いる.付録\ref{sec:セリフ}に家電A〜Fの発話内容を示す.
\end{itemize}



\subsection{手順}
\label{sec:how}
実験は十分な広さのある雑音のない部屋で,以下の手順で行う.5.〜8.を行う手法の順番は,順序効果を打ち消すためにカウンタバランスをとる.
\begin{enumerate}
	\item 被験者に各手法の説明を行う
	\item 被験者にタスクを説明する
	\item 被験者に家電の発話の特徴と配置場所を記憶させる
	\item 被験者に家電の操作方法を説明する
	\item 被験者に家電の操作を行わせる(\ref{sec:操作}に詳述)
	\item 家電の操作を1個行うごとに,それに連携した家電が発話するため,家電が発話した順に家電名とその家電の電源の状態を被験者に答えさせる(\ref{sec:回答}に詳述)
	\item 指示書(付録\ref{sec:指示})のタスクリストごとに5.〜6.を繰り返させる
	\item 被験者に使用した手法についてのアンケート(付録\ref{sec:nasaanke})に答えさせる
	\item 各手法ごとに,5.〜8.を繰り返させる
	\item 全手法の使用終了後に,被験者に全体を通してのアンケート(付録\ref{sec:after})に答えさせる
\end{enumerate}

\subsection{家電の操作}
\label{sec:操作}
\ref{sec:how}の手順5.では,家電に連携を起こさせて発話させるために被験者に家電の操作を行わせる.ここでは,実験者が指示したときに,指示書に書かれた指示を被験者に1個実行させる.
その後,被験者が操作を行った家電に連携している家電が順に発話する.

例えば,実験者が指示書のタスクリスト1の1.を行うように指示した場合,指示書に「家電Aの電源をONにする」という指示が記載されているため,この指示の通りに被験者が「家電Aの電源をONにする」とマイクに向けて発話することで,家電AがONになったときに連携する家電が順に発話する.
付録Eに実際の実験中の家電の動作を示す.

被験者は指示書に書かれたタスクリストの指示を1.〜9.まで行う.
それらが終わったら異なる手法でタスクリストの指示を行う.
\subsection{発話した家電の判別}
\label{sec:回答}
家電の操作終了後,被験者にはビープ音が流れたら,発話した家電の家電名と状態を口頭で回答させる.
1回の操作につき,発話する家電の数は2〜4個である.
それに先立ち,タスクの開始前に被験者には周囲にある家電の音声を聞かせて,家電の種類と発話の特徴を把握させる.

\subsection{環境}
実験環境を図\ref{fig:haichi}に示す.被験者の周囲に,家電を模したスピーカを配置する.
また,スピーカの制御をするためのPC,被験者の音声入力のためのマイク,被験者の様子を撮影して分析するためのビデオカメラを配置する.
先行システムで難しいとされていた,家電が同じ方向に存在している場合でも家電の判別が行えることを確かめるために,家電を被験者の前後2方向に配置している.
家電名はA〜Fとする.家電名を冷蔵庫などの実際に存在する家電名にしなかったのは,家電に複数の連携を行わせるときに,ユーザにとって不自然な連携にならないようにするためである.

\begin{figure}[p]
  \centering
  \includegraphics[width=14cm]{haichi.png}
     \caption{実験環境}
        \label{fig:haichi}
\end{figure}
\subsection{被験者}
本大学の大学生および大学院生の計12名である.

\subsection{評価尺度}
評価尺度は以下の通りである.
\begin{itemize}
	\item 家電の判別率\\
	被験者が家電の状態を正確に判別できているかを確かめる.
	\item 家電の判別にかかる時間\\
	全ての連携している家電が発話し終えてから発話した家電名を被験者が答え終わるまでの時間を計測する.
	\item 作業負荷\\
	手法ごとに被験者の家電の判別時の負荷を計測するために日本語版NASA-TLX\cite{nasa}を用いる.ただし,アンケート項目は被験者が理解しやすいように書き換えてアンケートを行う.実際に用いたアンケートを付録\ref{sec:nasaanke}に示す.
	\item 主観評価\\
	全タスク終了後に被験者の主観評価のためのアンケート(付録\ref{sec:after})を行う.
\end{itemize}

\section{結果}
本実験では,各結果に対して有意水準α=0.05として手法と家電の連携の数で二要因分散分析を行った.

判別率において,全被験者の各プロトタイプにおける判別率の平均を図\ref{fig:ans}に示す.
分析の結果,手法bと手法dの間において手法bが有意に判別率が高い結果となったが,それ以外の間に有意差はなかった.
また,判別時間において,全被験者の各手法における連携数別の平均時間を図\ref{fig:time}に示す.
分析の結果,全ての手法間において有意差はなかった.次に,手法ごとに日本語版NASA-TLXを用いたアンケートを行い,尺度に重みをつけずに単純平均するRTLX(raw-TLX)法\cite{rnasa}に基づき全ての項目の値を合計し,全被験者の平均を出したものを図\ref{fig:nasa}に示す.
分析の結果,手法bと手法cの間では有意差はなかったが,それ以外の手法間には有意差が見られた.
また,被験者の主観評価を調べるためのアンケート結果を表\ref{tab:anke1},\ref{tab:anke2},\ref{tab:anke3}に示す. 

\begin{figure}[p]
  \centering
  \includegraphics[width=12cm]{seitou.png}
     \caption{判別率}
        \label{fig:ans}
\end{figure}
\begin{figure}[p]
  \centering
  \includegraphics[width=13cm]{time.png}
     \caption{判別時間}
        \label{fig:time}
\end{figure}
\begin{figure}[p]
  \centering
  \includegraphics[width=13cm]{NASA.png}
     \caption{日本語版NASA-TLXの平均値}
        \label{fig:nasa}
\end{figure}
\begin{table}[p]
  \begin{center}
    \caption{全プロトタイプ使用後のアンケート結果1}
    \centering 
    \begin{tabular}{c|c|c;{.4pt/1pt}c|c} \Hline
      質問番号 & 質問内容 & 手法 & 順位 & 選択人数 \\ \hline \hline
      &  &  & 1 & 4 \\ \cdashline{4-5}[.4pt/2pt]
      &  & a & 2 & 8\\ \cdashline{4-5}[.4pt/2pt]
      &  &  & 3 & 0\\ \cdashline{4-5}[.4pt/2pt]
      &  &  & 4 & 0\\ \cdashline{3-5}[.4pt/1pt]
      &  &  & 1 & 6\\ \cdashline{4-5}[.4pt/1pt]
      &  & b & 2 & 2\\ \cdashline{4-5}[.4pt/2pt]
      &  &  & 3 & 3\\ \cdashline{4-5}[.4pt/2pt]
      1& 使いやすかったと思った順に &  & 4 & 1\\ \cdashline{3-5}[.4pt/1pt]
      & 記号を並び替えてください &  & 1 & 2\\ \cdashline{4-5}[.4pt/2pt]
      &  & c & 2 & 2\\ \cdashline{4-5}[.4pt/2pt]
      &  &  & 3 & 6\\ \cdashline{4-5}[.4pt/2pt]
      &  &  & 4 & 2\\ \cdashline{3-5}[.4pt/1pt]
      &  &  & 1 & 0\\ \cdashline{4-5}[.4pt/2pt]
      &  & d & 2 & 0\\ \cdashline{4-5}[.4pt/2pt]
      &  &  & 3 & 3\\ \cdashline{4-5}[.4pt/2pt]
      &  &  & 4 & 9\\ \Hline
    \end{tabular}
    \label{tab:anke1}
  \end{center}
  
\end{table}

\begin{table}[tbp]
  \begin{center}
    \caption{全プロトタイプ使用後のアンケート結果2}
    \centering 
     \hspace*{-3em}  
    \begin{tabular}{c|c|c|c} \Hline
      質問番号 & 質問内容 & 選択肢 & 選択人数 \\ \hline \hline
       &  & 思わなかった & 0 \\ \cdashline{3-4}[.4pt/1pt]
      2 & 家電同士が対話するシステムを & どちらかというと思わなかった & 3 \\ \cdashline{3-4}[.4pt/1pt]
      & 使用したいと思いましたか & どちらかというと思った & 7 \\ \cdashline{3-4}[.4pt/1pt]
      &  & 思った & 2 \\  \Hline
    \end{tabular}
    \label{tab:anke2}
  \end{center}
  
\end{table}




\begin{table}[tbp]
  \begin{center}
    \caption{全プロトタイプ使用後のアンケート結果3}
    \centering
    \begin{tabular}{c|p{12cm}} \Hline
      質問2の回答 &           質問3の回答内容  \\ \hline \hline
      (c) & 便利そうだから. \\ \cdashline{1-2}[.4pt/1pt]
      (b) & どの家電が話しているのか分からなくなってくるのに加えて,どの家電がON,OFFになっているのかも分からなくなったため.また,家で家電が話しているのが少し気持ち悪いと思いました. \\ \cdashline{1-2}[.4pt/1pt]
      (c) & 家電同士が対話することでちゃんと家電が連携しているということがわかるし,もし故障したときにどの家電が故障したのかも確認できるから. \\ \cdashline{1-2}[.4pt/1pt]
      (d) & 声に反応して家電が動いてくれると,身体の不自由な人々でも簡単に操作することができ,便利であると思ったから. \\ \cdashline{1-2}[.4pt/1pt]
      (c) & 何か別の作業をしている間に,家電同士の会話を聞くだけで,何がONで何がOFFになっているかが分かるから. \\ \cdashline{1-2}[.4pt/1pt]
      (b) & 自動化は嬉しいし助かるが,家電の声を聞き取っていなければならない.聞き取る必要があるシステムだと,少し煩わしいかなと思いました. \\ \cdashline{1-2}[.4pt/1pt]
      (d) & 1つの操作で複数のことが実行されるのはすごく便利だと思ったから. \\ \cdashline{1-2}[.4pt/1pt]
      (c) & 家電にもそれぞれのキャラがあっても面白そうなので良いかなと思った. \\ \cdashline{1-2}[.4pt/1pt]
      (c) & どの家電に変化があったために,どの家電が連携しているかがすぐ理解できたので便利だと感じたから. \\ \cdashline{1-2}[.4pt/1pt]
      (c) & 一応,何が起動したかを確認はしておきたいが,いちいち動くのが面倒くさい時に便利だと思ったから. \\ \cdashline{1-2}[.4pt/1pt]
      (c) & 生活の中で必ず連携して欲しい場面があるので,そういうときのためにあると便利だと思った. \\ \cdashline{1-2}[.4pt/1pt]
      (b) & 連携して動いてくれるのはいいが,どれが作動したのか,はっきり正確には分かりづらいから. \\ \cdashline{1-2} \Hline
    \end{tabular}
    \label{tab:anke3}
  \end{center}
\end{table}

\section{考察}
\subsection{判別率}
図\ref{fig:ans}に示すように,手法bと手法dの間において有意差が認められた.
よって,先行手法に比べて提案手法は正確に発話した家電の判別が行えるといえる.
また,手法bの判別率は95\%を超えており,同じ方向に複数の家電が存在していても家電の判別が可能であるといえる.

一方で,手法cと手法dの間には有意差がなかったことから,家電名の発話の有無に比べ,口調および声色の違いが家電の判別に影響するといえる.
さらに,手法bと手法cの間で有意差がなかったことから,家電の発話に特徴を持たせることで家電名を発話する場合と同程度の判別を行えていることがわかる.
よって,家電の発話に特徴を持たせることで家電名を発話しない分,発話時間を短くしても判別率を維持できているといえるため,手法bは有用性があるといえる.

また,手法aと手法bを比較すると,手法aの方が発話内容の情報量が多いため手法aの方が判別率が高くなると想定されるが,図\ref{fig:ans}に示すように,手法aと手法bの間には有意差は見られなかった.
この原因は,手法aでは発話時に家電自身の名前と動作の原因となった家電の名前の2つを発話するため,連携の数が多い場合に多くの家電名が発話され,それによって被験者が家電を取り違えてしまったために,手法aの判別率が想定よりも低下したと考えられる.
\subsection{判別時間}
図\ref{fig:time}に示すように,全ての手法間で有意差は見られなかった.よって,先行手法に比べて提案手法の方が発話した家電の判別を早くできるとはいえない.
この原因は,判別時間の測定法と家電の発話方法の不整合であると考えられる.
本実験での判別時間の定義は,介入を行うのは家電が発話を終えてからであることから,家電の発話後からの経過時間と決めた.
しかし,家電の発話は動作を行う理由,動作の内容の順に発話するため,ユーザが家電の発話の途中ですでに家電の判別ができている場合でも,家電の状態を把握するために発話を最後まで聞かないと家電の状態の把握ができない.そのため,被験者は回答ができず差が出なかったと考えられる.

\subsection{アンケート}
図\ref{fig:nasa}に示すように,手法bと手法cの間では有意差はなかったが,それ以外の手法間には有意差が見られた.
手法bと手法dの間に有意差が見られたことから,提案手法は先行手法に比べてユーザの作業の負荷を小さくすることができたといえる.

一方,ユーザの作業負荷においては手法bより手法aの方が作業負荷が少ないという結果になったにも関わらず,表\ref{tab:anke1}に示したアンケートに対する回答結果では,提案手法である手法bが最も使いやすいと思った被験者の数が多かった.ただし,手法aは全ての被験者が1番目か2番目に使いやすかったと回答しており,手法bは3番目,4番目に使いやすかったと回答した被験者がいた.

このような結果になったのは,手法bが良くないと回答した被験者は全てcの方が使いやすいと回答していたことから,口調および声色の違いよりも家電名を発話させる方が確実に発話した家電を判別できると考えたためであると考えられる.このような被験者がいたものの,質問番号1の手法bと手法cの回答結果を比較すると,全体的には家電名のみを付け加えるよりも各家電に特徴を持たせる方が使いやすいと感じたことがわかるため,提案手法に有用性があるといえる.
\section{課題}
\subsection{変化が発生するまでの時間}
提案手法の目的は発話した家電をより早く正確に判別できるようにすることであったが,図\ref{fig:time}より判別時間の早さが正確に比較できない結果になってしまったため,ユーザが介入を行うかどうかを考える時間の猶予に余裕を持たせられたかはわからない.
そのため,適切な介入可能時間を求める必要がある.

介入可能時間が長すぎると,ユーザが介入を行わないと判断した場合に,家電が発話してから実際に動作を始めるまでの時間が長いと感じてしまうことが考えられる.
その場合,家電の動作を自動化しているにも関わらずユーザが自分で操作し直した方が早く操作でき,自動化するメリットを感じられなくなってしまうことが考えられる.
そのため,介入可能時間はユーザが十分に介入可能であり,かつ家電が実際に動作を開始するまでの時間が長すぎないようにする必要がある.


\subsection{発話の特徴付けの方法}
図\ref{fig:ans}より,手法dに比べて手法bの判別率は有意に向上したことから,発話に特徴を持たせることは有効である.
しかし,本実験の手法aと手法bでの家電の発話は予め録音した人の声を用いて行った.
本実験では家電の数が6個であり,家電の発話のバリエーションが多くはなかったため録音した声でも実現可能であったが,実生活において,ホームオートメーションにより連携している家電は6個よりも多いことが想定され,さらに,ユーザは家電を買い替えたり,新しく家電を買ったりするため,家電の数は一定とはいえない.
そのため,音声合成を応用し,声色の違いを自動で生成することにより,それを各家電の発話の特徴として持たせられるようにする手法を検討することも価値がある.


\chapter{結言} 
スマート家電との音声対話において,スマート家電の発話時にユーザがどの家電が発話したかを判別しにくいという問題がある.
そこで本報告書では,各家電に異なる口調および声色で発話させることで家電の発話に特徴を持たせ,ユーザに発話した家電を判別しやすく通知する手法を提案した.そして,提案手法の有用性を評価するための実験を行った.
実験では,提案手法と先行手法の他に,比較用の手法として,提案手法に加え家電名を発話させる手法,先行手法に加え家電名を発話させる手法を比較し,提案手法の有用性を評価した.その結果,以下のことがわかった.
\begin{itemize}
	\item 提案手法は先行手法に比べて発話した家電とその家電の状態を正確に判別できる
	\item 提案手法は先行手法に比べてユーザの作業の負荷を少なくすることができる
	\item 提案手法と先行手法で判別時間に差はない
	\item 名前の発話よりも発話特徴の違いが判別に有効である
\end{itemize}
判別時間が実験で比較した手法間で有意な差が出なかった要因としては,介入を行うのは家電が発話を終えてからであるため,家電の発話が終わってからの経過時間で判別時間を計測していたが,家電の発話の最後まで聞かなければ家電の状態まで判別できなかったことが挙げられる.

今後の課題としては,判別を早く行えるような発話内容の検討や,介入可能時間の検討が挙げられる.また,本実験で用いた家電の音声は予め人の声を録音して用いたが,音声合成を用いて家電の発話に特徴を持たせる手法も検討課題としてあげられる.

\acknowledgement % 謝辞
本研究を行うにあたり,研究課題の設定や研究に対する姿勢,本報告書の作成に
至るまで,全ての面で丁寧なご指導を頂きました,本学情報工学・人間科学系辻野嘉宏教
授,倉本到准教授,山本景子助教に厚く御礼申し上げます.
本報告書執筆にあたり貴重な助言を多数頂きました,本学情報工学専攻
岡田みなみ先輩,本学先端ファイブロ科学専攻吉田全孝先輩,本学情報工学課程 伊藤哲也氏,今村祥子氏を
はじめとする,人間情報技術研究室の皆さん,学生生活を通じて著者の支えとなった
家族や友人に深く感謝致します.

% 参考文献
\begin{thebibliography}{}
  \bibitem{Grinter} W.Keith Edwards and R.E. Grinter, ``At home with ubiquitous computing:seven challenges,'' Ubicomp 2001:Ubiquitous Computing,pp.256-272,Oct. 2001.
  \bibitem{TRON} 東京大学総合研究博物館,The TRON Project,(オンライン),入手先〈http://www.um.u-tokyo.ac.jp/DM\_CD/DM\_TECH/BTRON/PROJ/HOME.HTM〉(参照 2014-1-17).
  \bibitem{ECHO} ECHONET CONSORTIUM,エコーネット規格,(オンライン),入手先〈http://www.echonet.gr.jp/spec/index.htm〉(参照 2014-1-17).
  \bibitem{denshin} 日本電信電話株式会社,News Release 040308,(オンライン), 入手先〈http://www.ntt.co.jp/news/news04/0403/040308.html〉(参照 2014-2-04).
  \bibitem{Hitach} Hitach, Ltd.,HITACHI : ニュースリリース : ホームネットワーク技術を活用した生活サポート「ホラソネットワーク」事業を開始,(オンライン),入手先〈http://www.hitachi.co.jp/New/cnews/hl\_031211.html〉(参照 2014-2-04).
  \bibitem{sumi} 角田真大, ``スマート家電連携への自然な介入を可能にする音声対話手法,''卒業研究報告書,京都工芸繊維大学,Feb. 2014.
%  \bibitem{koe} 横山正人,井上和夫,``声質感覚に基づく個人性知覚情報の抽出,''Mar. 2010  
  \bibitem{bunseki} 橋本 誠,樋口宜男,``音声の個人性知覚に影響を及ぼす音響的特徴の分析,''電子情報通信学会技術研究報告,pp.29-36,Oct. 1996.
  \bibitem{seisei} 北村達也,``音声における個人性の知覚と生成について,''甲南大学紀要.知能情報学編,vol.1,no.2,pp.141-155,2008
  \bibitem{chara} 宮崎千明,平野 徹,東中竜一郎,牧野俊朗,松尾義博,佐藤理史,``文節機能部の確率的書き換えによる言語表現のキャラクタ性変換,''人工知能学会論文誌,vol.31,no.1,2016
  \bibitem{nasa} 芳賀 繁,水上直樹,``日本語版NASA-TLXによるメンタルロードワーク測定,''人間工学,vol.32,no.2,pp.71-79,1996.
  \bibitem{rnasa} J.C. Byers,A. Bittner,and S. Hill,``Traditional and raw task load index' (TLX) correlations: Are paired comparisons necessary,'' Advances in industrial ergonomics and safety I, pp.481-485,1989
  
\end{thebibliography}
\appendix % 付録

\chapter{実験で用いた家電の発話内容リスト}
\label{sec:セリフ}
○:A〜F
\begin{itemize}
	\item 家電A
	\begin{itemize}
		\item 家電Aです.
		\item 家電○の電源がON(OFF)になったので電源をON(OFF)にします.
	\end{itemize}

	\item 家電B
	\begin{itemize}
		\item 家電Bやで.
		\item 家電○の電源がON(OFF)になったから電源をON(OFF)にすんで.
	\end{itemize}

	
	\item 家電C
	\begin{itemize}
		\item 家電Cだよ.
		\item 家電○の電源がON(OFF)になったので電源をON(OFF)にするよ.
	\end{itemize}

		
	\item 家電D
	\begin{itemize}
	\item 家電Dでっせ.
		\item 家電○の電源がON(OFF)になったから電源をON(OFF)にしまっせ.
	\end{itemize}


	\item 家電E
	\begin{itemize}
		\item 家電Eですよ.
		\item 家電○の電源がON(OFF)になったから電源をON(OFF)にしますね.
	\end{itemize}

	\item 家電F
	\begin{itemize}
		\item 家電Fじゃ.
		\item 家電○の電源がON(OFF)になっとるけえ電源をON(OFF)にしとくよ.
	\end{itemize}
\end{itemize}

\newpage

\chapter{実験で被験者に渡した指示書}
\label{sec:指示}
\begin{multicols}{2}
\subsection*{タスクリスト1}
	\begin{enumerate}
		\item 家電Aの電源をONにする
		\item 家電Fの電源をONにする
		\item 家電Eの電源をOFFにする
		\item 家電Bの電源をOFFにする
		\item 家電Cの電源をONにする
		\item 家電Dの電源をOFFにする
		\item 家電Aの電源をOFFにする
		\item 家電Fの電源をOFFにする		
		\item 家電Eの電源をONにする
	\end{enumerate}
\subsection*{タスクリスト2}
	\begin{enumerate}
		\item 家電Cの電源をONにする
		\item 家電Dの電源をONにする
		\item 家電Bの電源をOFFにする
		\item 家電Eの電源をOFFにする
		\item 家電Aの電源をONにする
		\item 家電Fの電源をOFFにする
		\item 家電Cの電源をOFFにする
		\item 家電Dの電源をOFFにする
		\item 家電Bの電源をONにする
	\end{enumerate}
\end{multicols}

\begin{multicols}{2}
\subsection*{タスクリスト3}
	\begin{enumerate}
		\item 家電Eの電源をONにする
		\item 家電Bの電源をONにする	
		\item 家電Cの電源をOFFにする
		\item 家電Fの電源をOFFにする
		\item 家電Dの電源をONにする
		\item 家電Aの電源をOFFにする
		\item 家電Eの電源をOFFにする
		\item 家電Bの電源をOFFにする
		\item 家電Cの電源をONにする
	\end{enumerate}
\subsection*{タスクリスト4}
	\begin{enumerate}
		\item 家電Fの電源をONにする
		\item 家電Aの電源をONにする
		\item 家電Dの電源をOFFにする
		\item 家電Cの電源をOFFにする
		\item 家電Bの電源をONにする
		\item 家電Eの電源をOFFにする
		\item 家電Fの電源をOFFにする
		\item 家電Aの電源をOFFにする
		\item 家電Dの電源をONにする
	\end{enumerate}
\end{multicols}	


\newpage
\chapter{各プロトタイプ使用ごとに行うアンケート}
\label{sec:nasaanke}
{\small
\begin{itemize}
	\item 精神的要求(0:低い,100:高い)
	\begin{itemize}
		\item タスクを実行中に,家電の電源を切り替える,家電の発話内容を聞いてどの家電が話したか判断する,ことにどれくらいの知覚的活動が必要だったと感じましたか
	\end{itemize}
\begin{center}
\unitlength=1cm
\begin{picture}(10, 1)(-5, -.5)
\put(-5, 0){\line(1,0){10}}
\put(0, -.1){\line(0,1){.2}}

\multiput(-5, -.1)(1, 0){11}{\line(0,1){.2}}
\put(-6, -.3){\makebox(0, 0){$低い$}}
\put(-5, -.3){\makebox(0, 0){$0$}}
\put(-4, -.3){\makebox(0, 0){$10$}}
\put(-3, -.3){\makebox(0, 0){$20$}}
\put(-2, -.3){\makebox(0, 0){$30$}}
\put(-1, -.3){\makebox(0, 0){$40$}}
\put(0, -.3){\makebox(0, 0){$50$}}
\put(1, -.3){\makebox(0, 0){$60$}}
\put(2, -.3){\makebox(0, 0){$70$}}
\put(3, -.3){\makebox(0, 0){$80$}}
\put(4, -.3){\makebox(0, 0){$90$}}
\put(5, -.3){\makebox(0, 0){$100$}}
\put(6, -.3){\makebox(0, 0){$高い$}}
\end{picture}
\end{center}
	\item 身体的要求(0:低い,100:高い)
	\begin{itemize}
		\item タスクを実行中に,家電の電源を切り替える,音声を聞く,ことにどれくらいの身体的活動が必要だったと感じましたか
	\end{itemize}		
\begin{center}
\unitlength=1cm
\begin{picture}(10, 1)(-5, -.5)
\put(-5, 0){\line(1,0){10}}
\put(0, -.1){\line(0,1){.2}}

\multiput(-5, -.1)(1, 0){11}{\line(0,1){.2}}
\put(-6, -.3){\makebox(0, 0){$低い$}}
\put(-5, -.3){\makebox(0, 0){$0$}}
\put(-4, -.3){\makebox(0, 0){$10$}}
\put(-3, -.3){\makebox(0, 0){$20$}}
\put(-2, -.3){\makebox(0, 0){$30$}}
\put(-1, -.3){\makebox(0, 0){$40$}}
\put(0, -.3){\makebox(0, 0){$50$}}
\put(1, -.3){\makebox(0, 0){$60$}}
\put(2, -.3){\makebox(0, 0){$70$}}
\put(3, -.3){\makebox(0, 0){$80$}}
\put(4, -.3){\makebox(0, 0){$90$}}
\put(5, -.3){\makebox(0, 0){$100$}}
\put(6, -.3){\makebox(0, 0){$高い$}}
\end{picture}
\end{center}
	\item 忙しさ(0:低い,100:高い)
	\begin{itemize}
		\item タスクを実行するにあたって,家電の発話の間隔とスピードからくる時間的圧力はどの程度だったと思いますか
	\end{itemize}
\begin{center}
\unitlength=1cm
\begin{picture}(10, 1)(-5, -.5)
\put(-5, 0){\line(1,0){10}}
\put(0, -.1){\line(0,1){.2}}

\multiput(-5, -.1)(1, 0){11}{\line(0,1){.2}}
\put(-6, -.3){\makebox(0, 0){$低い$}}
\put(-5, -.3){\makebox(0, 0){$0$}}
\put(-4, -.3){\makebox(0, 0){$10$}}
\put(-3, -.3){\makebox(0, 0){$20$}}
\put(-2, -.3){\makebox(0, 0){$30$}}
\put(-1, -.3){\makebox(0, 0){$40$}}
\put(0, -.3){\makebox(0, 0){$50$}}
\put(1, -.3){\makebox(0, 0){$60$}}
\put(2, -.3){\makebox(0, 0){$70$}}
\put(3, -.3){\makebox(0, 0){$80$}}
\put(4, -.3){\makebox(0, 0){$90$}}
\put(5, -.3){\makebox(0, 0){$100$}}
\put(6, -.3){\makebox(0, 0){$高い$}}
\end{picture}
\end{center}
	\item 努力(0:低い,100:高い)
	\begin{itemize}
		\item 家電の発話内容を聞いてどの家電が話したか正確に判断するのにどの程度がんばったと思いますか
	\end{itemize}		
\begin{center}
\unitlength=1cm
\begin{picture}(10, 1)(-5, -.5)
\put(-5, 0){\line(1,0){10}}
\put(0, -.1){\line(0,1){.2}}

\multiput(-5, -.1)(1, 0){11}{\line(0,1){.2}}
\put(-6, -.3){\makebox(0, 0){$低い$}}
\put(-5, -.3){\makebox(0, 0){$0$}}
\put(-4, -.3){\makebox(0, 0){$10$}}
\put(-3, -.3){\makebox(0, 0){$20$}}
\put(-2, -.3){\makebox(0, 0){$30$}}
\put(-1, -.3){\makebox(0, 0){$40$}}
\put(0, -.3){\makebox(0, 0){$50$}}
\put(1, -.3){\makebox(0, 0){$60$}}
\put(2, -.3){\makebox(0, 0){$70$}}
\put(3, -.3){\makebox(0, 0){$80$}}
\put(4, -.3){\makebox(0, 0){$90$}}
\put(5, -.3){\makebox(0, 0){$100$}}
\put(6, -.3){\makebox(0, 0){$高い$}}
\end{picture}
\end{center}
	\item 達成度(0:低い,100:高い)
	\begin{itemize}
		\item 家電の発話内容を聞いてどの家電が話したか正確に判断することについて,あなたはどの程度成功したと思いますか
	\end{itemize}
\begin{center}
\unitlength=1cm
\begin{picture}(10, 1)(-5, -.5)
\put(-5, 0){\line(1,0){10}}
\put(0, -.1){\line(0,1){.2}}

\multiput(-5, -.1)(1, 0){11}{\line(0,1){.2}}
\put(-6, -.3){\makebox(0, 0){$低い$}}
\put(-5, -.3){\makebox(0, 0){$0$}}
\put(-4, -.3){\makebox(0, 0){$10$}}
\put(-3, -.3){\makebox(0, 0){$20$}}
\put(-2, -.3){\makebox(0, 0){$30$}}
\put(-1, -.3){\makebox(0, 0){$40$}}
\put(0, -.3){\makebox(0, 0){$50$}}
\put(1, -.3){\makebox(0, 0){$60$}}
\put(2, -.3){\makebox(0, 0){$70$}}
\put(3, -.3){\makebox(0, 0){$80$}}
\put(4, -.3){\makebox(0, 0){$90$}}
\put(5, -.3){\makebox(0, 0){$100$}}
\put(6, -.3){\makebox(0, 0){$高い$}}
\end{picture}
\end{center}
	\item 不満度(0:低い,100:高い)
	\begin{itemize}
		\item タスクを実行中に,いらいら,不安,落胆, ストレス,悩み等をどの程度感じましたか
	\end{itemize}
\begin{center}
\unitlength=1cm
\begin{picture}(10, 1)(-5, -.5)
\put(-5, 0){\line(1,0){10}}
\put(0, -.1){\line(0,1){.2}}

\multiput(-5, -.1)(1, 0){11}{\line(0,1){.2}}
\put(-6, -.3){\makebox(0, 0){$低い$}}
\put(-5, -.3){\makebox(0, 0){$0$}}
\put(-4, -.3){\makebox(0, 0){$10$}}
\put(-3, -.3){\makebox(0, 0){$20$}}
\put(-2, -.3){\makebox(0, 0){$30$}}
\put(-1, -.3){\makebox(0, 0){$40$}}
\put(0, -.3){\makebox(0, 0){$50$}}
\put(1, -.3){\makebox(0, 0){$60$}}
\put(2, -.3){\makebox(0, 0){$70$}}
\put(3, -.3){\makebox(0, 0){$80$}}
\put(4, -.3){\makebox(0, 0){$90$}}
\put(5, -.3){\makebox(0, 0){$100$}}
\put(6, -.3){\makebox(0, 0){$高い$}}
\end{picture}
\end{center}
\end{itemize}

}

\newpage
\chapter{全プロトタイプ使用後に行うアンケート}
\label{sec:after}
\begin{enumerate}
	\item 使いやすかったと思った順に記号を並び替えてください
	\begin{enumerate}
		\item[(1)] 1回目
		\item[(2)] 2回目
		\item[(3)] 3回目
		\item[(4)] 4回目
	\end{enumerate}
\begin{figure}[H]
  \centering
  \includegraphics[width=8cm]{anke.png}
\end{figure}
	\item 家電同士が対話するシステムを使用したいと思いましたか?
	\begin{enumerate}
		\item 思わなかった
		\item どちらかというと思わなかった
		\item どちらかというと思った
		\item 思った
	\end{enumerate}
		\item それは何故ですか?
\begin{figure}[H]
  \centering
  \includegraphics[width=8cm]{anke2.png}
\end{figure}
\end{enumerate}


\newpage
\chapter{実験中の家電の動作}
\label{sec:action}
\subsection*{タスクリスト1}
\begin{multicols}{2}
	\begin{enumerate}
		\item 家電Aの電源がONになる
		\begin{enumerate}
			\item 家電Aが発話する(ON)
			\item 家電Bが発話する(ON)
		\end{enumerate}

		\item 家電Fの電源がONになる
		\begin{enumerate}
			\item 家電Fが発話する(ON)		
			\item 家電Bが発話する(OFF)
			\item 家電Eが発話する(ON)
			\item 家電Cが発話する(ON)
		\end{enumerate}	
			
		\item 家電Eの電源がOFFになる
		\begin{enumerate}
			\item 家電Eが発話する(OFF)
			\item 家電Bが発話する(ON)
			\item 家電Fが発話する(OFF)
		\end{enumerate}
	
		\item 家電Bの電源がOFFになる
		\begin{enumerate}	
			\item 家電Bが発話する(OFF)	
			\item 家電Cが発話する(OFF)
			\item 家電Aが発話する(OFF)
		\end{enumerate}
		
		\item 家電Cの電源がONになる
		\begin{enumerate}
			\item 家電Cが発話する(ON)	
			\item 家電Dが発話する(ON)
		\end{enumerate}	
		
		\item 家電Dの電源がOFFになる
		\begin{enumerate}
			\item 家電Dが発話する(OFF)
			\item 家電Aが発話する(ON)
			\item 家電Cが発話する(OFF)
			\item 家電Bが発話する(ON)	
		\end{enumerate}	
		
		\item 家電Aの電源がOFFになる
		\begin{enumerate}
			\item 家電Aが発話する(OFF)	
			\item 家電Eが発話する(ON)
			\item 家電Dが発話する(ON)
			\item 家電Fが発話する(ON)
		\end{enumerate}
		\item 家電Fの電源がOFFになる
		\begin{enumerate}
			\item 家電Fが発話する(OFF)
			\item 家電Dが発話する(OFF)
			\item 家電Eが発話する(OFF)
		\end{enumerate}
		
		\item 家電Eの電源がONになる
		\begin{enumerate}
			\item 家電Eが発話する(ON)
			\item 家電Fが発話する(ON)
		\end{enumerate}
	\end{enumerate}
\end{multicols}
\newpage
\subsection*{タスクリスト2}
\begin{multicols}{2}
	\begin{enumerate}
		\item 家電Cの電源がONになる
		\begin{enumerate}
			\item 家電Cが発話する(ON)
			\item 家電Eが発話する(ON)
		\end{enumerate}
		
		\item 家電Dの電源がONになる
		\begin{enumerate}
			\item 家電Dが発話する(ON)
			\item 家電Eが発話する(OFF)
			\item 家電Bが発話する(ON)
			\item 家電Aが発話する(ON)
		\end{enumerate}	
			
		\item 家電Bの電源がOFFになる
		\begin{enumerate}
			\item 家電Bが発話する(OFF)
			\item 家電Eが発話する(ON)
			\item 家電Dが発話する(OFF)
		\end{enumerate}
	
		\item 家電Eの電源がOFFになる
		\begin{enumerate}
			\item 家電Eが発話する(OFF)		
			\item 家電Aが発話する(OFF)
			\item 家電Cが発話する(OFF)
		\end{enumerate}
		
		\item 家電Aの電源がONになる
		\begin{enumerate}
			\item 家電Aが発話する(ON)
			\item 家電Fが発話する(ON)
		\end{enumerate}	
		
		\item 家電Fの電源がOFFになる
		\begin{enumerate}
			\item 家電Fが発話する(OFF)
			\item 家電Cが発話する(ON)
			\item 家電Aが発話する(OFF)
			\item 家電Eが発話する(ON)	
		\end{enumerate}	
		
		\item 家電Cの電源がOFFになる
		\begin{enumerate}	
			\item 家電Cが発話する(OFF)
			\item 家電Bが発話する(ON)
			\item 家電Fが発話する(ON)
			\item 家電Dが発話する(ON)
		\end{enumerate}
		\item 家電Dの電源がOFFになる
		\begin{enumerate}
			\item 家電Dが発話する(OFF)
			\item 家電Fが発話する(OFF)
			\item 家電Bが発話する(OFF)
		\end{enumerate}
		
		\item 家電Bの電源がONになる
		\begin{enumerate}
			\item 家電Bが発話する(ON)
			\item 家電Dが発話する(ON)
		\end{enumerate}
	\end{enumerate}
\end{multicols}
\newpage

\subsection*{タスクリスト3}
\begin{multicols}{2}
	\begin{enumerate}
		\item 家電Eの電源がONになる
		\begin{enumerate}
			\item 家電Eが発話する(ON)
			\item 家電Fが発話する(ON)
		\end{enumerate}
		
		\item 家電Bの電源がONになる
		\begin{enumerate}
			\item 家電Bが発話する(ON)
			\item 家電Fが発話する(OFF)
			\item 家電Cが発話する(ON)
			\item 家電Dが発話する(ON)
		\end{enumerate}	
			
		\item 家電Cの電源がOFFになる
		\begin{enumerate}
			\item 家電Cが発話する(OFF)
			\item 家電Fが発話する(ON)
			\item 家電Bが発話する(OFF)
		\end{enumerate}
	
		\item 家電Fの電源がOFFになる
		\begin{enumerate}	
			\item 家電Fが発話する(OFF)	
			\item 家電Dが発話する(OFF)
			\item 家電Eが発話する(OFF)
		\end{enumerate}
		
		\item 家電Dの電源がONになる
		\begin{enumerate}
			\item 家電Dが発話する(ON)
			\item 家電Aが発話する(ON)
		\end{enumerate}	
		
		\item 家電Aの電源がOFFになる
		\begin{enumerate}
			\item 家電Aが発話する(OFF)
			\item 家電Eが発話する(ON)
			\item 家電Dが発話する(OFF)
			\item 家電Fが発話する(ON)	
		\end{enumerate}	
		
		\item 家電Eの電源がOFFになる
		\begin{enumerate}	
			\item 家電Eが発話する(OFF)
			\item 家電Cが発話する(ON)
			\item 家電Aが発話する(ON)
			\item 家電Bが発話する(ON)
		\end{enumerate}
		\item 家電Bの電源がOFFになる
		\begin{enumerate}
			\item 家電Bが発話する(OFF)
			\item 家電Aが発話する(OFF)
			\item 家電Cが発話する(OFF)
		\end{enumerate}
		
		\item 家電Cの電源がONになる
		\begin{enumerate}
			\item 家電Cが発話する(ON)
			\item 家電Bが発話する(ON)
		\end{enumerate}
	\end{enumerate}
\end{multicols}	
\newpage
\subsection*{タスクリスト4}
\begin{multicols}{2}
	\begin{enumerate}
		\item 家電Fの電源がONになる
		\begin{enumerate}
			\item 家電Fが発話する(ON)
			\item 家電Cが発話する(ON)
		\end{enumerate}
		
		\item 家電Aの電源がONになる
		\begin{enumerate}
			\item 家電Aが発話する(ON)
			\item 家電Cが発話する(OFF)
			\item 家電Dが発話する(ON)
			\item 家電Bが発話する(ON)
		\end{enumerate}	
			
		\item 家電Dの電源がOFFになる
		\begin{enumerate}
			\item 家電Dが発話する(OFF)
			\item 家電Cが発話する(ON)
			\item 家電Aが発話する(OFF)
		\end{enumerate}
	
		\item 家電Cの電源がOFFになる
		\begin{enumerate}	
			\item 家電Cが発話する(OFF)	
			\item 家電Bが発話する(OFF)
			\item 家電Fが発話する(OFF)
		\end{enumerate}
		
		\item 家電Bの電源がONになる
		\begin{enumerate}	
			\item 家電Bが発話する(ON)
			\item 家電Eが発話する(ON)
		\end{enumerate}	
		
		\item 家電Eの電源がOFFになる
		\begin{enumerate}
			\item 家電Eが発話する(OFF)
			\item 家電Fが発話する(ON)
			\item 家電Bが発話する(OFF)
			\item 家電Cが発話する(ON)	
		\end{enumerate}	
		
		\item 家電Fの電源がOFFになる
		\begin{enumerate}	
			\item 家電Fが発話する(OFF)
			\item 家電Dが発話する(ON)
			\item 家電Eが発話する(ON)
			\item 家電Aが発話する(ON)
		\end{enumerate}
		\item 家電Aの電源がOFFになる
		\begin{enumerate}
			\item 家電Aが発話する(OFF)
			\item 家電Eが発話する(OFF)
			\item 家電Dが発話する(OFF)
		\end{enumerate}
		
		\item 家電Dの電源がONになる
		\begin{enumerate}
			\item 家電Dが発話する(ON)
			\item 家電Aが発話する(ON)
		\end{enumerate}
	\end{enumerate}
\end{multicols}	







\end{document}
